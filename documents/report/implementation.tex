In this section, we discuss the two homemade implementations of the algorithm used in this project: PageRank and Louvain community detection.

PageRank is an iterative process that can potentially last forever. As a stopping condition we put a number of iterations as well as an error threshold. If the error does not exceed a certain threshold over one iteration, then the algorithm stops. If it does not, then the algorithm runs over a fixed number of iteration. This strategy is very common in the various implementations and can be seen in \citetitle{hagbergExploringNetworkStructure2008}. This made it quite convenient to compare and benchmark our implementation. As a result, we discovered that our own implementation yields the same result but is much less optimised in terms of performance. Indeed, \citetitle{hagbergExploringNetworkStructure2008}'s implementation is more than 100 times faster and takes more than 400 times less RAM as it can be seen on the following table. This result is clearly surprising but only because the in-house implementation uses a dense matrix instead of a sparse matrix. This has the effect of using less RAM and significantly speeding up the calculations. With the use of a sparse matrix in our implementation, the computation takes 2s with a RAM consumption of about 0.2G. Other optimisations could then be made but the density of the matrix is really the core issue here. This is mainly explained by the Power Law distribution. Since a lot of vertices have very few edges, a lot of rows in the adjacency matrix have a lot of zeros. With a sparse matrix, we avoid storing all of these zeros and can therefore considerably speed up the computation.


\begin{table}[ht!]
\centering
\begin{tabular}{|l|l|l|l|l|} 
\hline
implementation & nodes & edges & time  & RAM     \\ 
\hline
\citetitle{hagbergExploringNetworkStructure2008}       & 15537 & 58150 & 0.78s & 0.128G  \\ 
\hline
homemade (dense)       & 15537 & 58150 & 91.21s   & 5.81G   \\
\hline
homemade (sparse)      & 15537 & 58150 & 2.18s   & 0.215G   \\
\hline
\end{tabular}
\caption{Benchmarks of PageRank algorithm on Fedora Linux 36 with Intel i7-8550U (8) @ 4.000GHz and 16GB of RAM}
\end{table}

Louvain is an iterative algorithm that works in series of 2 steps until convergence. The first one is to assign every node to a community based on the modularity gain. The second step consists in building a new graph by transforming the communities into hypernodes.
When implementing Louvain method, we took a small graph to ensure that the implementation was correct. However, when we tried to apply it on the graph based on \textit{Deep Link No Merge}, our implementation was too slow and we stopped the process after 10min without having a result.
When comparing with the implementation of \citetitle{hagbergExploringNetworkStructure2008}, we see that it is much faster and ends in a few seconds with the graph \textit{Deep Link No Merge}. Indeed, they do some optimizations and in particular they precalculate the gain of modularity. The computation of modularity gain is the most called instruction in the algorithm and it seems that it is the main reason why our implementation is too slow. Therefore, we used only a subset of the full graph to yield some results. Again, it shows significant difference bewtween our and \citetitle{hagbergExploringNetworkStructure2008}'s implementation. In any case, it is a successful implementation and still useful to understand the flow of the algorithm. On the other hand, other trials and errors may have been encountered during the project. This is the subject of the next section.

\begin{table}[ht!]
\centering
\begin{tabular}{|l|l|l|l|l|} 
\hline
implementation & nodes & edges & time  & RAM     \\ 
\hline
\citetitle{hagbergExploringNetworkStructure2008}       & 1000 & 4852 & 0.37s & 3.478MB  \\ 
\hline
homemade      & 1000 & 4852 & 18.76s   & 3.235MB   \\
\hline
\end{tabular}
\caption{Benchmarks of Louvain algorithm on macOS Monterey with Apple M1 and 8Go of RAM}
\end{table}

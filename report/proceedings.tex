\documentclass{sigchi}

% Load basic packages
\usepackage{balance}       % to better equalize the last page
\usepackage{graphics}      % for EPS, load graphicx instead 
\usepackage[T1]{fontenc}   % for umlauts and other diaeresis
\usepackage{txfonts}
\usepackage{mathptmx}
\usepackage[pdflang={en-US},pdftex]{hyperref}
\usepackage{color}
\usepackage{booktabs}
\usepackage{textcomp}
\usepackage{tikz}
\usetikzlibrary{shapes, arrows}


% Some optional stuff you might like/need.
\usepackage{microtype}        % Improved Tracking and Kerning
% \usepackage[all]{hypcap}    % Fixes bug in hyperref caption linking
\usepackage{ccicons}          % Cite your images correctly!
\usepackage[utf8]{inputenc} % for a UTF8 editor only

% If you want to use todo notes, marginpars etc. during creation of
% your draft document, you have to enable the "chi_draft" option for
% the document class. To do this, change the very first line to:
% "\documentclass[chi_draft]{sigchi}". You can then place todo notes
% by using the "\todo{...}"  command. Make sure to disable the draft
% option again before submitting your final document.
\usepackage{todonotes}

% Paper metadata (use plain text, for PDF inclusion and later
% re-using, if desired).  Use \emtpyauthor when submitting for review
% so you remain anonymous.
\def\plaintitle{Reddit cryptocurrency comments analysis}
\def\plainauthor{First Author, Second Author, Third Author,
  Fourth Author, Fifth Author, Sixth Author}
\def\emptyauthor{}
\def\plainkeywords{Reddit; Cryptocurrency.}
% \def\plaingeneralterms{Documentation, Standardization}

% llt: Define a global style for URLs, rather that the default one
\makeatletter
\def\url@leostyle{%
  \@ifundefined{selectfont}{
    \def\UrlFont{\sf}
  }{
    \def\UrlFont{\small\bf\ttfamily}
  }}
\makeatother
\urlstyle{leo}

% To make various LaTeX processors do the right thing with page size.
\def\pprw{8.5in}
\def\pprh{11in}
\special{papersize=\pprw,\pprh}
\setlength{\paperwidth}{\pprw}
\setlength{\paperheight}{\pprh}
\setlength{\pdfpagewidth}{\pprw}
\setlength{\pdfpageheight}{\pprh}

% Make sure hyperref comes last of your loaded packages, to give it a
% fighting chance of not being over-written, since its job is to
% redefine many LaTeX commands.
\definecolor{linkColor}{RGB}{6,125,233}
\hypersetup{%
  pdftitle={\plaintitle},
% Use \plainauthor for final version.
%  pdfauthor={\plainauthor},
  pdfauthor={\emptyauthor},
  pdfkeywords={\plainkeywords},
  pdfdisplaydoctitle=true, % For Accessibility
  bookmarksnumbered,
  pdfstartview={FitH},
  colorlinks,
  citecolor=black,
  filecolor=black,
  linkcolor=black,
  urlcolor=linkColor,
  breaklinks=true,
  hypertexnames=false
}

\usepackage[inline]{enumitem}
\usepackage{tasks}



% create a shortcut to typeset table headings
% \newcommand\tabhead[1]{\small\textbf{#1}}

% End of preamble. Here it comes the document.
\begin{document}

\title{\plaintitle}

\numberofauthors{3}
\author{%
  \alignauthor{Albin Aliu\\
    % \affaddr{for Submission}\\
    % \affaddr{University of Fribourg}\\
    \email{\href{mailto:albin.aliu@unifr.ch}{\color{black}albin.aliu@unifr.ch}}}\\
  \alignauthor{François-Xavier Wicht\\
    % \affaddr{for Submission}\\
    % \affaddr{University of Fribourg}\\
    \email{\href{mailto:francois-xavier.wicht@unifr.ch}{\color{black}francois-xavier.wicht@unifr.ch}}}\\
      \alignauthor{Grégoire Rebstein\\
    % \affaddr{for Submission}\\
    % \affaddr{University of Fribourg}\\
    \email{\href{mailto:gregoire.rebstein@unifr.ch}{\color{black}gregoire.rebstein@unifr.ch}}}
}

\maketitle

\begin{abstract}
Lorem ipsum dolor sit amet, consectetur adipiscing elit. Aliquam vulputate, eros nec fringilla commodo, arcu libero tristique magna, sed placerat mauris nisl ac quam. Sed sed purus ac eros viverra pharetra ac eget lacus. Suspendisse laoreet condimentum lorem sit amet bibendum. Morbi volutpat lectus vitae massa porttitor sollicitudin. Sed maximus ipsum vel est feugiat luctus. Sed suscipit lorem et est scelerisque gravida. Sed sed leo ac enim feugiat bibendum eu eget purus. Maecenas eget sollicitudin mi. Proin hendrerit luctus leo, ut cursus augue tempor ac. 

\end{abstract}


% ACM Classfication

% \begin{CCSXML}
%   <ccs2012>
%   <concept>
%   <concept_id>10003120.10003121.10003122.10003334</concept_id>
%   <concept_desc>Human-centered computing~User studies</concept_desc>
%   <concept_significance>500</concept_significance>
%   </concept>
%   <concept>
%   <concept_id>10003120.10003130</concept_id>
%   <concept_desc>Human-centered computing~Collaborative and social computing</concept_desc>
%   <concept_significance>300</concept_significance>
%   </concept>
%   </ccs2012>
% \end{CCSXML}

% \ccsdesc[500]{Human-centered computing~User studies}
% \ccsdesc[300]{Human-centered computing~Collaborative and social computing}

% \ccsdesc[500]{Human-centered computing~Human computer interaction (HCI)}
% \ccsdesc[100]{Human-centered computing~User studies}

% Author Keywords
\keywords{\plainkeywords}

% Print the classification codes
% \printccsdesc
% Please use the 2012 Classifiers and see this link to embed them in the text: \url{https://dl.acm.org/ccs/ccs_flat.cfm}
\section{Introduction}
Since the rise of cryptocurrencies, a lot of research have attempted to predict their price rates \cite{kooPredictionBitcoinPrice2021,coccoPredictionsBitcoinPrices2021,liMultiwindowBitcoinPrice2021}. Due to the transparency of cryptocurrency transactions, researchers have considered using transaction information such as overall trends and cyclical changes to predict cryptocurrency prices \cite{liCrossCryptocurrencyRelationship2022}.

In this project, we attempt to find a link between the activity in cryptocurrency subreddits and the price of Bitcoin. For that purpose, we scraped data directly from said subreddits, structured data in a network architecture and performed several analysis such as PageRank, Louvain community detection and correlation analysis.

This report is structured the following way. Firstly, we discuss the scraping methodology we used, secondly the data models we tried and considered for the analysis and thirdly we outline the analysis of our model. As a closing word, we highlight trials and errors during this project.

\section{Related works}
\input{related_works}
\section{Conclusion}
In conclusion, during this work we have collected data from 10 cryptocurrency-related subreddits and have performed analysis on them. We first have modeled the data and have concluded that the one that best fits our need was the Deep Link No Merge model. We pursued with the analysis and proved that our data followed a Power Law distribution. With help of PageRank, we showed that the discussion on these subreddits tend to have a lasting interest. Furthermore, with Louvain, we showed that each community detected spread over multiple subreddits. However, the majority of each community has a foothold in one unique subreddit. Correlation between subreddits activity and Bitcoin price movement was later revealed with a value of $-0.85$. We also found that the more days were analysed the better results it yielded. All in all, the results that were produced in this work are consistent with related ones. Finally, a dedicated work could be done on this subject to further discover relations and relevant results.


\newpage
\bibliographystyle{SIGCHI-Reference-Format}
\small
\bibliography{references}
\newpage
% \printbibliography

\end{document}

%%% Local Variables:
%%% mode: latex
%%% TeX-master: t
%%% End:
